\documentclass[a4paper,11pt,twoside]{article}
\usepackage{GUD_conf_template}
\setcounter{section}{0}
\setcounter{figure}{0}
\setcounter{table}{0}
\setcounter{equation}{0}
\setcounter{secnumdepth}{1}
\setcounter{secnumdepth}{1}
\begin{document}
% ================ Начало статьи ================
% 
\authors {% автор статьи
А.~С.~Гудков
}%конец списка авторов

\topic{%начало заголовка статьи
Современные алгоритмы компьютерных шахмат
}%конец заголовка статьи
% 
% ================ Аннотация ================
\annotation{%% начало текста аннотации
В работе рассматривается создание шахматной игры и реализация её определённого алгоритма поиска наилучших ходов с использованием языка программирования JavaScript.
}% конец аннотации

\begin{multicols}{2}
% ================ Начало основного текста ================
% 
% ================ Введение ================
\section*{% заголовок
Введение}\vspace{-10pt}% начало введеня

Компьютерные шахматы – популярный термин из области исследования искусственного интеллекта, означающий создание специального программного обеспечения для игры в шахматы. У шахмат довольно простые правила. Две противоборствующие стороны, шесть разновидностей фигур и одна цель – дать мат сопернику.
% конец текста введеня
% 
% ================ 1 раздел статьи ================
\section{% заголовок 1 раздела статьи
Принцип работы и создание игры
}\vspace{-10pt}% начало текста 1 раздела статьи
Создание шахматной программы включает несколько этапов:

\textbf{–} визуализация доски;

\textbf{–} перемещение фигур;

\textbf{–} оценка доски;

\textbf{–} дерево поиска.

Первые два этапа легко реализовываются с помощью JavaScript фреймворка Vue.js. Поэтому перейдем к самому шахматному алгоритму, который включает два последних пункта.

Общее количество уникальных партий в шахматы превышает количество атомов во Вселенной, отчего им не грозит быть посчитанными полностью. Поэтому в бой вступают алгоритмы оценки позиций и деревьев возможных ходов.

Чтобы узнать, у какой стороны преимущество в том или ином положении следует оценивать шахматную доску. Самый простой путь – подсчитать относительную силу фигур на ней, присвоив каждой фигуре ее стоимость (пешка – 1, конь и офицер – 3, тура – 5, ферзь – 9, а король – 900, т. к. бесценен). Для белых цена фигур положительная, для черных – отрицательная. Теперь при помощи оценки доски алгоритм может выбирать ход с максимальным преимуществом.

Исходная матрица ходов представлена на рисунке 1.

\image{[width=0.8\columnwidth]{GUD_image-1.png}\vspace{-10pt}
\caption{%------Подпись под рисунком ---------
Матрица ходов
}}\vspace{-10pt}
% ----- конец добавления рисунка

Для того чтобы составить конкуренцию человеку алгоритм должен уметь «видеть» на несколько ходов вперёд. На данном этапе мы создаём дерево поиска, анализирующее все возможные ходы до заданной глубины, и после на его листьях происходит оценка доски.

Далее мы возвращаем значение оценки потомка в родительский узел. Выбор оценки зависит от того, ход какой стороны сейчас просчитывается. Если ход чёрных, то он минимизируется (т.к. фигуры чёрных отрицательны), если белых, то соответственно максимизируется.

С деревом поиска алгоритм начинает понимать базовую тактику шахмат и уже способен не только составить конкуренцию, но и обыграть большинство игроков. Стоит отметить, что эффективность поиска увеличивается с его глубиной, но также возрастает и время. 
% конец текста 1 раздела статьи
% 
% ================ Выводы ================
\section{Выводы}\vspace{-10pt}
В ходе работы была реализована шахматная программа и её определённый алгоритм поиска наилучшего хода. Описанный алгоритм является основой современных компьютерных шахмат.

Созданная игра доступна как для компьютеров, так и для смартфонов по ссылке: \textcolor{blue}{\url{https://warrior-coder.github.io/Chess-BOT}}. Просмотреть исходный код можно на персональной странице GitHub: \textcolor{blue}{\url{https://github.com/warrior-coder/Chess-BOT}}.

\end{multicols}
% ================ Оформление данных об авторах ================
%
\authorFIO{Гудков Алексей Сергеевич}
\authorAbout{
студент 1 курса факультета информационных технологий и управления БГУИР, gudkou\_fitu@mail.ru.
}

\authorFIO{Научный руководитель: Навроцкий Анатолий Александрович}
\authorAbout{
заведующий кафедрой информационных технологий автоматизированных систем БГУИР, кандидат физико-математических наук, доцент, navrotsky@bsuir.by.

}
% конец данных об авторах
%
%  ================ Конец материалов ================
\end{document}